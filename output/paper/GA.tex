\documentclass{article}
\usepackage{amsmath}
\usepackage{algorithm}
\usepackage{algorithmic}
\usepackage{graphicx}
\usepackage{geometry}
\geometry{a4paper,margin=2cm}
\usepackage[utf8]{inputenc}
\usepackage{ctex}

\title{基于遗传算法的烟幕干扰弹投放策略优化}
\author{}
\date{}

\begin{document}

\maketitle

\section{遗传算法总体设计}

本研究采用遗传算法(Genetic Algorithm, GA)解决烟幕干扰弹投放策略优化问题。该算法旨在最大化多个导弹的总遮蔽时间,通过优化无人机飞行参数和干扰弹投放策略,实现最优的防御效果。

\subsection{问题编码}

每个个体(candidate solution)表示一个完整的投放策略,编码结构如下:

\begin{align}
Individual = \{DroneID_1: [v_x, v_y, [(t_1, d_1), (t_2, d_2), ...]], ...DroneID_n\}
\end{align}

其中:
\begin{itemize}
    \item $v_x, v_y$:无人机在x、y方向的速度分量(m/s)
    \item $t_i$:第i个干扰弹的投放时间(s)
    \item $d_i$:第i个干扰弹的烟雾起爆延迟(s)
    \item 速度约束:$70 \leq \sqrt{v_x^2 + v_y^2} \leq 140$ m/s
\end{itemize}

\section{核心算法组件}

\subsection{初始化策略}

个体初始化函数\texttt{create\_individual()}采用随机生成策略:

\begin{algorithm}[H]
\caption{个体初始化算法}
\begin{algorithmic}[1]
\FOR{each $drone\_id$ in drone\_ids}
    \STATE 随机生成速度角度 $\theta \in [0, 2\pi)$
    \STATE 随机生成速度大小 $v \in [70, 140]$
    \STATE 计算速度分量:$v_x = v \cos(\theta)$, $v_y = v \sin(\theta)$
    \FOR{$i = 1$ to $n\_jammers$}
        \STATE 生成投放时间:$t_i \in [0, 10]$
        \STATE 生成起爆延迟:$d_i \in [0.1, 10]$
    \ENDFOR
\ENDFOR
\end{algorithmic}
\end{algorithm}

\subsection{适应度评估}

适应度函数\texttt{evaluate\_individual()}计算个体的总遮蔽时间:

\begin{align}
fitness(individual) = \sum_{missile\_id \in targeted\_missiles} CoverTime(missile\_id)
\end{align}

评估流程:
\begin{enumerate}
    \item 重置所有无人机的干扰弹配置
    \item 根据个体参数配置无人机速度和干扰弹投放策略
    \item 调用物理仿真系统计算各导弹的遮蔽时间
    \item 返回总遮蔽时间作为适应度值
\end{enumerate}

\subsection{交叉算子}

采用混合交叉策略\texttt{crossover()}:

\textbf{速度交叉:}
\begin{align}
v_{child} = \alpha \cdot v_{parent1} + (1-\alpha) \cdot v_{parent2}
\end{align}
其中$\alpha \in [0.3, 0.7]$为随机权重。

\textbf{干扰弹参数交叉:}
对每个干扰弹参数$(t_i, d_i)$,随机选择来自parent1或parent2的参数对。

\subsection{变异算子}

实现自适应变异策略\texttt{adaptive\_mutate()}:

\begin{align}
mutation\_rate = base\_rate \cdot (1 + stagnation\_factor)
\end{align}

\begin{align}
noise\_scale = initial\_scale \cdot \exp(-decay\_factor \cdot generation)
\end{align}

变异操作:
\begin{itemize}
    \item \textbf{速度变异:}添加高斯噪声后应用速度约束
    \item \textbf{投放时间变异:}在$[0, 10]$范围内添加噪声
    \item \textbf{起爆延迟变异:}在$[0.1, 10]$范围内添加噪声
\end{itemize}

\subsection{选择策略}

采用锦标赛选择\texttt{tournament\_selection()}:

\begin{algorithm}[H]
\caption{锦标赛选择算法}
\begin{algorithmic}[1]
\STATE 随机选择$tournament\_size$个个体
\STATE 返回其中适应度最高的个体
\end{algorithmic}
\end{algorithm}

\subsection{种群多样性保持}

\subsubsection{多样性计算}
基于速度向量的欧几里得距离计算种群多样性:

\begin{align}
Diversity = \frac{1}{N(N-1)/2} \sum_{i<j} \sqrt{\sum_{drone}(v_{i,drone} - v_{j,drone})^2}
\end{align}

\subsubsection{种群重启机制}
当算法停滞时(连续多代无改进),触发种群重启:
\begin{itemize}
    \item 保留当前最佳个体
    \item 重新随机初始化其余个体
    \item 重置停滞计数器
\end{itemize}

\section{优化算法流程}

\begin{algorithm}[H]
\caption{主优化算法}
\begin{algorithmic}[1]
\STATE 初始化种群$P_0$
\STATE $best\_fitness = 0$, $stagnation\_count = 0$
\FOR{$generation = 1$ to $max\_generations$}
    \STATE 评估所有个体适应度
    \STATE 更新最佳个体和适应度
    \IF{$current\_best > best\_fitness$}
        \STATE $best\_fitness = current\_best$
        \STATE $stagnation\_count = 0$
    \ELSE
        \STATE $stagnation\_count++$
    \ENDIF
    
    \IF{$stagnation\_count > threshold$}
        \STATE 执行种群重启
    \ENDIF
    
    \STATE 计算种群多样性
    
    \FOR{$i = 1$ to $population\_size$}
        \STATE $parent1 = tournament\_selection(P)$
        \STATE $parent2 = tournament\_selection(P)$
        \STATE $child = crossover(parent1, parent2)$
        \STATE $child = adaptive\_mutate(child, generation)$
        \STATE $child = repair\_individual(child)$
        \STATE $P_{new}[i] = child$
    \ENDFOR
    
    \STATE 精英保留:保持最佳个体到新种群
    \STATE $P = P_{new}$
\ENDFOR
\RETURN 最佳个体及其适应度
\end{algorithmic}
\end{algorithm}

\section{约束处理}

\subsection{速度约束}
使用\texttt{apply\_velocity\_constraints()}函数确保速度在$[70, 140]$ m/s范围内:

\begin{align}
v_{new} = \begin{cases}
\frac{70}{\|v\|} \cdot v & \text{if } \|v\| < 70 \\
\frac{140}{\|v\|} \cdot v & \text{if } \|v\| > 140 \\
v & \text{otherwise}
\end{cases}
\end{align}

\subsection{时间约束}
\texttt{repair\_jammers\_timing()}函数确保同一无人机的干扰弹投放时间间隔不少于1秒:

\begin{align}
t_i = \max(t_i, t_{i-1} + 1.0) \quad \text{for sorted } t_i
\end{align}

\section{算法参数设置}

\begin{table}[h]
\centering
\begin{tabular}{|l|l|}
\hline
参数 & 值 \\
\hline
种群大小 & 100 \\
最大代数 & 500 \\
锦标赛大小 & 3 \\
基础变异率 & 0.1 \\
精英保留比例 & 0.1 \\
停滞阈值 & 50代 \\
速度范围 & [70, 140] m/s \\
投放时间范围 & [0, 10] s \\
起爆延迟范围 & [0.1, 10] s \\
\hline
\end{tabular}
\caption{遗传算法参数设置}
\end{table}

\section{性能优化}

\subsection{数值计算优化}
使用Numba的\texttt{@njit}装饰器加速核心数值计算:
\begin{itemize}
    \item 多样性计算函数
    \item 速度约束应用
    \item 时间修复算法
    \item 交叉和变异的数值操作
\end{itemize}

\subsection{收敛性分析}
算法提供收敛可视化功能,实时监控:
\begin{itemize}
    \item 最佳适应度变化曲线
    \item 平均适应度变化曲线
    \item 种群多样性变化曲线
\end{itemize}

\section{算法特点与创新}

\begin{enumerate}
    \item \textbf{多目标适应度:}同时优化多个导弹的遮蔽效果
    \item \textbf{自适应机制:}变异率和噪声强度随代数自适应调整
    \item \textbf{多样性保持:}通过种群重启防止早熟收敛
    \item \textbf{约束修复:}确保所有解满足物理和工程约束
    \item \textbf{精英策略:}保证算法收敛性和解的质量
    \item \textbf{高效实现:}结合Numba加速和并行化设计
\end{enumerate}

\section{结果输出}

算法提供多种结果输出格式:
\begin{itemize}
    \item 文本格式详细报告(包含各导弹遮蔽时间间隔)
    \item Excel表格(标准化投放策略参数)
    \item 收敛过程可视化图表
\end{itemize}

该遗传算法通过精心设计的编码方式、约束处理机制和自适应策略,能够有效求解复杂的烟幕干扰弹投放优化问题,为实际防御系统提供可行的解决方案。

\end{document}